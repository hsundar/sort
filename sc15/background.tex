\section{Background}
Keep related work here and separate from introduction. some glue text here before we go into separate subsections on the energy aspects and the algorithmic aspects of sorting.

\subsection{energy considerations on GPUs}

\subsection{Sorting Algorithms}
The design and implementation of parallel algorithms for sorting is a well studied subject. In our discussion of prior work, we focus our attention to distributed memory algorithms that have been experimentally shown to scale to large core counts and large datasets. We will also discuss related sorting algorithms (albeit not at scale) that study the energy-efficiency of sorting. We begin by a formal definition of distributed sorting.


Given an array $A$ with $N$ keys and an order (comparison) relation, we would like to sort the elements of $A$ in ascending order. In a distributed memory machine with $p$ tasks, every task is assigned an $N/p$-sized block of $A$. Upon completion, task $i$ will have the $i^{\text{th}}$ block of the sorted array. Note that the overall ordering of the keys is determined by the ordering of the tasks as well as the ordering of the keys on each task.  

\paragraph{Distributed sorting}

The most popular algorithm used in actual distributed
sort implementations is SampleSort, originally proposed by Frazer and McKellar\cite{samplesort}. Given $p$ tasks, we
reshuffle the elements of $A$ in $p$ buckets so that the all the
keys in the $i^\text{th}$ bucket are smaller (or equal) than the keys in
the $(i+1)^\text{th}$ bucket. Bucket $i$ is assigned to task $i$ and thus,
once we have shuffled $A$, each task can sort its bucket of
keys in an embarrassingly parallel manner using any local
(shared-memory-parallel) sort algorithm. The challenge is to obtain good load-balancing, i.e., ensuring that each 
task  has roughly the same number of keys, while minimizing communication costs.  

One possible way to reshuffle $A$ is to estimate the boundaries for each bucket, by selecting $p-1$ keys, which we call
``splitters''. This can be done  by sampling a
subset of keys in $A$, sorting them and selecting splitters from that set. Once these $p-1$ splitters have been
selected, a global data exchange takes place to move the original keys of every task to their correct bucket.
An additional local sort is invoked to finalize the output array. SampleSort is well understood. However, its performance is quite sensitive to the selection of splitters, which can result in load imbalance. Most importantly, the final data exchange---requiring $\mathcal{O}(p^2)$ messages---can congest the network. As a result SampleSort may scale 
suboptimally, especially when the communication volume approaches the available hardware limits \cite{hyksort}.

Parallel HistogramSort~\cite{kale93,solomonik10} is
a variant of SampleSort that efficiently estimates the splitters. The authors presented
one of the largest distributed comparison sort runs (32K cores on
BG/P) with $46\%$ efficiency. The algorithm overlaps communication
and computation in all stages. During the splitter estimation
stage, the iterative estimation of the ranks of the splitters is
combined with partial local sorting of the data, by using the
splitter candidates as pivots for quicksort. Once the splitters
are estimated, the communication of data is staged and overlapped
with local merging. In~\cite{solomonik10} the best throughput was
obtained on 16,384 cores of Jaguar XT4 at ORNL for 8M (64-bit) keys
per core; the sort took 4.3 seconds achieving a in-RAM throughput of
14.4TB/min. 

CloudRAMSort \cite{kim12} demonstrated good scalability on 256 nodes with shared memory
parallelism using {\texttt pThreads} and SIMD vectorization. The
best results are for sorting 1TB of data (10byte key + 90byte
record) in 4.6 secs achieving an in-RAM throughput of 12.6 TB/min. They
use a variant of Histogram Sort~\cite{kale93}, where the
samples are iteratively computed (in parallel) in order to guarantee
a minimum quality of load-balance. Additionally, the communication of the records is split into two parts by first communicating the keys followed by the values overlapped with the merging of the keys. 

In recent work \cite{hyksort}, we have addressed the scalability of sorting on very large clusters. Hyksort\cite{hyksort} is an extension of quicksort on a Hypercube \cite{wagar87} to a $k$-ary Hypercube along with an efficient parallel (median) selection algorithm. 

\paragraph{Energy-efficient and GPU sorting}
mention the general approaches. 

\paragraph{Code variant tuning}
Apart from Nitro~\cite{muralidharan:2014},
several programmer-directed autotuning frameworks support
tuning of code/algorithmic variants.
Petabricks~\cite{PetaBricks} 
supports user specification of \textit{transforms} that
are analogous to functions. Transforms are automatically
composed together to form hybrid algorithms using a compiler
framework and an adaptive algorithm~\cite{increa11}. Petabricks, however,
implicitly tunes variants for the size of the input data set. Ding et
al.~\cite{ding:2015} propose extensions to the PetaBricks language to
enable support for tuning based on arbitrary input features.

\paragraph{Multi-objective optimization}
