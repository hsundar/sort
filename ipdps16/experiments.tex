\section{Experimental Methodology}

This section describes the input data for sort and the
three target architectures used in our experiment.

\paragraph{Input Data}
As the performance of sort is dependent on its input data, we 
use a variety of data types, distributions and sizes in our experiments.
We consider XXX data types: 
integer, long integer, float, double, {\color{red} fill in
the rest.}
The distribution of key values 
includes a uniform distribution, a Gaussian distribution, 
{\color{red} fill in
the rest.}
Sizes are motivated by each experiment and the capacity of the
target architecture.  {\color{red} For Titan, ...  
For K20c ... For Jetson ...  Perhaps this goes in the results section.}

\paragraph{Titan (ORNL)} 
We are interested in 
achieving scalable energy efficiency for distributed
sorting on supercomputers, and therefore the target architecture is the
Titan system at Oak Ridge.  Titan has 
{\color{red}... CPUs ... GPUs ...  compiler/OpenMP/MPI installations?}
We would like to measure performance on the GPUs on Titan, but encountered
an obstacle in the software installation.  
The CUDA version used on Titan is CUDA XXX, which is out-of-date with respect to
sorting implementations we are using.  We are also unable to measure
power and energy in the same way on Titan as on the other platforms.
Therefore, we will focus our
Titan experiments on using the OpenMP implementation of sorting, and focus
data gathering on the communication and scaling aspects of our optimizations.
We will then extrapolate
Titan GPU results using other representative clusters.

%\noindent
%{\bf K20 cluster.}
%{\color{red} This is the SCI cluster.  Omit if we don't have results on it.}

\paragraph{NVIDIA K80 Standalone GPU}
For the node-level experiments, we used an NVIDIA K80 (Kepler 
generation) standalone GPU.
It has 26 GPU streaming multiprocessors, for a total of 4992 cores, 
and 24 GBytes of memory.

The K80 supports 25 core clock frequency settings ranging from 562 Mhz to 875 Mhz
in 13 Mhz increments.
It supports two memory frequencies as well, 
but we do not adjust memory frequency in these experiments 
because the lower memory frequency of 324 Mhz
is far lower than peak of 2505 Mhz
and is therefore going to perform poorly in a bandwidth-limited
algorithm such as sort.
For frequency adjustments and energy and power measurements, we use the NVML
API~\cite{} from NVIDIA. The K80 also uses a mechanism called \textit{AutoBoost} to dynamically
vary clock frequencies to fill available power headroom. We disable this feature
for more consistent data collection.

\paragraph{Jetson TK1 Cluster}
We also measured performance of distributed sorting on an NVIDIA 
Jetson TK1 cluster; the nodes of the Jetson are low-power and lightweight,
consisting of a single GPU streaming multiprocessor (Kepler generation)
with 192 cores, and 
four-plus-one ARM cores, where the fifth ARM core is used as a master processor.
The nodes have a unified DRAM of 2 GBytes, which is shared between CPUs and
GPUs, and separate cache structures for CPU and GPU.
The cluster we use in this experiment has 16 nodes, and
the nodes are connected with {\color{red} ... network details.}
The software installation uses CUDA 6.5, nvcc compiler version 6.5.35,
MPI version 1.6.5, and OpenMP version 3.1.

The power and energy reported for
Jetson are physical measurements  
using the Yokogawa WT310 digital multimeter.
We measure the voltage drop across a known precision resistance in series with the device under test (DUT).
With a known resistance and measured voltage on that resistance,
the current can be obtained with the equation $I=V/R$.
Here, the resistance is 0.020 Ohms with a 1\% variation.
To determine the power, we use the equation $P=IV$, where $I$ is the value calculated above, and $V$ is 12.19V.
The Jetson has 14 core clock frequencies ranging from 72MHz to 852MHz,
and twelve memory frequencies from 12.75MHz to 924MHz;
since collecting physical measurements 
on all 1728 combinations of core/memory frequency per data set would 
be prohibitively time-consuming, we only varied core frequencies for our experiments.

While not capable of the high GPU performance of the K80
since it has only one-thirteenth of the SMs,
the Jetson cluster looks to the future of high-performance and 
embedded GPU platforms. Support for unified memory allows us to look 
at power and energy without the data movement required to copy
from CPU to GPU, and the large number of frequency adjustments
allow us to examine how the large number of degrees of freedom
in energy management impacts energy, power and performance in 
code variant selection. 
