\section{Energy and Power-Efficient Node Sorting}
The previously-described distributed sorting algorithm relies
on a node-level sort of the subset of data assigned to each node at 
each stage of the sort.  This section describes how we arrive 
at a node-level sorting algorithm that is both energy/power-efficient and
high performance.  

\subsection{Overview of Approach}
We first present an overview of the node-level sorting approach, 
looking at factors such as resource selection in a heterogeneous platform, 
sorting algorithms, and how we extend an existing framework for performance
optimization to also take into account energy and power consumption.

\paragraph{GPU vs. Parallel CPU Sorting}
{\color{red}Establish that GPU is highest performing and most energy efficient for node sort.
Show performance (and possibly energy) 
difference for both Titan and Jetson compared with OpenMP?}  

\paragraph{Sorting algorithms as code variants}
It is well established that the best algorithm for sorting
is dependent on input data set (type, size and distribution), target 
architecture and implementation details. Therefore, the best
implementation cannot be determined until
run-time without prior knowledge of these factors.
We refer to different sorting algorithms or implementations
as \emph{code variants}.  
A number of techniques for code variant selection and algorithm
selection have been described in the literature, and our approach
will rely on recent advancements in this area as described in this section.
 
In this paper, we select among two algorithms to use for
the node-level sort.  
\begin{itemize}
\item \emph{Merge Sort:}
Merge sort sorts a list of data by 
recursively splitting the list in half, sorting each half,
and then merging the two sorted lists together.
The Merge Sort implementation we use is
part of the ModernGPU~\cite{modernGPU} library of GPU
primitives.  
\item \emph{Radix Sort:}
Radix sort achieves a sorted list by grouping keys by individual digits  
that have the same position and value.
The radix sort implementation is provided in CUB~\cite{cub}. 
\end{itemize}
%{\color{red} May want to also say when one might be preferable to another.}

\paragraph{Managing energy and power on the GPU} 
We use two mechanisms to adjust energy and power usage on the GPU.  
We can monitor energy or power usage for each of the two 
sort algorithms, and together with performance measurements,
select the preferred algorithm.  In addition, the target NVIDIA GPUs
allow adjusting of the clock frequency, or frequency of the memory
bus.  Through monitoring energy or power at different frequency
settings, we can select the preferred frequency(ies).  

Since we would prefer an implementation that is both high performing and
power or energy efficient, we must develop a \emph{selection 
criteria} that considers multiple optimization goals in selecting the
node-level sorting algorithm.
%employ multi-objective tuning, which selects an 
%implementation according to both its performance and its energy usage or
%peak power requirements.  
The next subsection will describe
a number of different selection criteria we explore in this paper
and their overall impact on performance, energy and power.

\paragraph{Code variant selection using Nitro}
The system described in this paper used for code variant
selection extends
the Nitro autotuning framework~\cite{muralidharan:2014}.
Nitro provides a library interface that permits expert programmers to
express code variants along with meta-information that aids
the system in selecting among the set of variants at runtime.
Figure~\ref{fig:overview} illustrates the approach in Nitro.  
A learning algorithm -- Support Vector Machine (SVM) classifier by default -- constructs a 
code variant selection
model on the target architecture as a result of an offline training phase on the
 same architecture.
For each architecture, training data has the form
$\{(\mathbf{x}_1, y_1), \hdots, (\mathbf{x}_M, y_M)\}$, where
each $\mathbf{x}_i$ represents an input feature vector and each
$y_i$ represents the best variant for that input.
When presented with a new, unseen input at runtime, the model
predicts the best variant to use. 
%For sort, prior work has used data type, data set size and presortedness
%as features~\cite{muralidharan:2014}.
In this paper, we use the number of input elements and the size
of the data type (in bits) as features.
These two features are available when the
sort is invoked, and can be used in consulting a model for code
variant selection at runtime.
In this paper, we extend Nitro in two ways: (1) we treat 
different clock frequencies as code variants, in addition to the
different sorting algorithms; and, (2) the model is trained
using both performance and energy/power data, according to the 
selection criteria outlined in the next subsection.

\subsection{Code Variant Selection Criteria}
Application tuning that looks at multiple optimization criteria
is referred to as \emph{multi-objective tuning}.
A challenge with multi-objective tuning is that the solution must encompass
a tradeoff space between different optimization objectives.  Any 
solution among the \emph{Pareto frontier} is valid; these points are ones
for which there is no other solution that has better metrics among
all the set of objectives.
{\color{red} CLEAN UP AND ADD CITATIONS!!!
Techniques for multi-objective tuning resolve this selection in several
ways:
(1) treat one objective as independent (e.g., an equivalence class
as in PetaBricks paper);  (2) weight one objective above others; 
(3) ask users; (4) use heuristics to drive optimization; or, (5) come up with combined metric.}

In this paper, we have chosen to use a set of fairly standard combined metrics.
We explore which leads to the best reduction in energy or power with the least
impact on performance.  These metrics are as follows: 
\begin{itemize}
\item MKeys per Joule: {\color{red} Define}
\item MKeys$^2$ per Joule: {\color{red} Define}
\item MKeys per Watt: {\color{red} Define}
\item MKeys$^2$ per Watt: {\color{red} Define}
\end{itemize}
These were selected because they capture the relationship between throughput
and energy or power.  Further, it is straightforward to build a model 
for code variant selection by consolidating on a single metric.
In our experiments we will show {\color{red} some indication that this works well.}
